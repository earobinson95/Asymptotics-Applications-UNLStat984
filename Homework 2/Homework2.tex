\documentclass[12pt,]{article}
\usepackage{lmodern}
\usepackage{amssymb,amsmath}
\usepackage{ifxetex,ifluatex}
\usepackage{fixltx2e} % provides \textsubscript
\ifnum 0\ifxetex 1\fi\ifluatex 1\fi=0 % if pdftex
  \usepackage[T1]{fontenc}
  \usepackage[utf8]{inputenc}
\else % if luatex or xelatex
  \ifxetex
    \usepackage{mathspec}
  \else
    \usepackage{fontspec}
  \fi
  \defaultfontfeatures{Ligatures=TeX,Scale=MatchLowercase}
\fi
% use upquote if available, for straight quotes in verbatim environments
\IfFileExists{upquote.sty}{\usepackage{upquote}}{}
% use microtype if available
\IfFileExists{microtype.sty}{%
\usepackage{microtype}
\UseMicrotypeSet[protrusion]{basicmath} % disable protrusion for tt fonts
}{}
\usepackage[margin=1in]{geometry}
\usepackage{hyperref}
\hypersetup{unicode=true,
            pdftitle={Homework 2},
            pdfauthor={Emily Robinson},
            pdfborder={0 0 0},
            breaklinks=true}
\urlstyle{same}  % don't use monospace font for urls
\usepackage{color}
\usepackage{fancyvrb}
\newcommand{\VerbBar}{|}
\newcommand{\VERB}{\Verb[commandchars=\\\{\}]}
\DefineVerbatimEnvironment{Highlighting}{Verbatim}{commandchars=\\\{\}}
% Add ',fontsize=\small' for more characters per line
\usepackage{framed}
\definecolor{shadecolor}{RGB}{248,248,248}
\newenvironment{Shaded}{\begin{snugshade}}{\end{snugshade}}
\newcommand{\AlertTok}[1]{\textcolor[rgb]{0.94,0.16,0.16}{#1}}
\newcommand{\AnnotationTok}[1]{\textcolor[rgb]{0.56,0.35,0.01}{\textbf{\textit{#1}}}}
\newcommand{\AttributeTok}[1]{\textcolor[rgb]{0.77,0.63,0.00}{#1}}
\newcommand{\BaseNTok}[1]{\textcolor[rgb]{0.00,0.00,0.81}{#1}}
\newcommand{\BuiltInTok}[1]{#1}
\newcommand{\CharTok}[1]{\textcolor[rgb]{0.31,0.60,0.02}{#1}}
\newcommand{\CommentTok}[1]{\textcolor[rgb]{0.56,0.35,0.01}{\textit{#1}}}
\newcommand{\CommentVarTok}[1]{\textcolor[rgb]{0.56,0.35,0.01}{\textbf{\textit{#1}}}}
\newcommand{\ConstantTok}[1]{\textcolor[rgb]{0.00,0.00,0.00}{#1}}
\newcommand{\ControlFlowTok}[1]{\textcolor[rgb]{0.13,0.29,0.53}{\textbf{#1}}}
\newcommand{\DataTypeTok}[1]{\textcolor[rgb]{0.13,0.29,0.53}{#1}}
\newcommand{\DecValTok}[1]{\textcolor[rgb]{0.00,0.00,0.81}{#1}}
\newcommand{\DocumentationTok}[1]{\textcolor[rgb]{0.56,0.35,0.01}{\textbf{\textit{#1}}}}
\newcommand{\ErrorTok}[1]{\textcolor[rgb]{0.64,0.00,0.00}{\textbf{#1}}}
\newcommand{\ExtensionTok}[1]{#1}
\newcommand{\FloatTok}[1]{\textcolor[rgb]{0.00,0.00,0.81}{#1}}
\newcommand{\FunctionTok}[1]{\textcolor[rgb]{0.00,0.00,0.00}{#1}}
\newcommand{\ImportTok}[1]{#1}
\newcommand{\InformationTok}[1]{\textcolor[rgb]{0.56,0.35,0.01}{\textbf{\textit{#1}}}}
\newcommand{\KeywordTok}[1]{\textcolor[rgb]{0.13,0.29,0.53}{\textbf{#1}}}
\newcommand{\NormalTok}[1]{#1}
\newcommand{\OperatorTok}[1]{\textcolor[rgb]{0.81,0.36,0.00}{\textbf{#1}}}
\newcommand{\OtherTok}[1]{\textcolor[rgb]{0.56,0.35,0.01}{#1}}
\newcommand{\PreprocessorTok}[1]{\textcolor[rgb]{0.56,0.35,0.01}{\textit{#1}}}
\newcommand{\RegionMarkerTok}[1]{#1}
\newcommand{\SpecialCharTok}[1]{\textcolor[rgb]{0.00,0.00,0.00}{#1}}
\newcommand{\SpecialStringTok}[1]{\textcolor[rgb]{0.31,0.60,0.02}{#1}}
\newcommand{\StringTok}[1]{\textcolor[rgb]{0.31,0.60,0.02}{#1}}
\newcommand{\VariableTok}[1]{\textcolor[rgb]{0.00,0.00,0.00}{#1}}
\newcommand{\VerbatimStringTok}[1]{\textcolor[rgb]{0.31,0.60,0.02}{#1}}
\newcommand{\WarningTok}[1]{\textcolor[rgb]{0.56,0.35,0.01}{\textbf{\textit{#1}}}}
\usepackage{longtable,booktabs}
\usepackage{graphicx,grffile}
\makeatletter
\def\maxwidth{\ifdim\Gin@nat@width>\linewidth\linewidth\else\Gin@nat@width\fi}
\def\maxheight{\ifdim\Gin@nat@height>\textheight\textheight\else\Gin@nat@height\fi}
\makeatother
% Scale images if necessary, so that they will not overflow the page
% margins by default, and it is still possible to overwrite the defaults
% using explicit options in \includegraphics[width, height, ...]{}
\setkeys{Gin}{width=\maxwidth,height=\maxheight,keepaspectratio}
\IfFileExists{parskip.sty}{%
\usepackage{parskip}
}{% else
\setlength{\parindent}{0pt}
\setlength{\parskip}{6pt plus 2pt minus 1pt}
}
\setlength{\emergencystretch}{3em}  % prevent overfull lines
\providecommand{\tightlist}{%
  \setlength{\itemsep}{0pt}\setlength{\parskip}{0pt}}
\setcounter{secnumdepth}{0}
% Redefines (sub)paragraphs to behave more like sections
\ifx\paragraph\undefined\else
\let\oldparagraph\paragraph
\renewcommand{\paragraph}[1]{\oldparagraph{#1}\mbox{}}
\fi
\ifx\subparagraph\undefined\else
\let\oldsubparagraph\subparagraph
\renewcommand{\subparagraph}[1]{\oldsubparagraph{#1}\mbox{}}
\fi

%%% Use protect on footnotes to avoid problems with footnotes in titles
\let\rmarkdownfootnote\footnote%
\def\footnote{\protect\rmarkdownfootnote}

%%% Change title format to be more compact
\usepackage{titling}

% Create subtitle command for use in maketitle
\providecommand{\subtitle}[1]{
  \posttitle{
    \begin{center}\large#1\end{center}
    }
}

\setlength{\droptitle}{-2em}

  \title{Homework 2}
    \pretitle{\vspace{\droptitle}\centering\huge}
  \posttitle{\par}
  \subtitle{STAT 984}
  \author{Emily Robinson}
    \preauthor{\centering\large\emph}
  \postauthor{\par}
      \predate{\centering\large\emph}
  \postdate{\par}
    \date{September 13, 2019}

\usepackage{amsmath}
\usepackage{amssymb}
\usepackage{amsthm}

\begin{document}
\maketitle

\hypertarget{exercise-1.21}{%
\subsubsection{Exercise 1.21}\label{exercise-1.21}}

Let \(x_1, ..., x_n\) be a simple random sample from an exponential
distribution with density \(f(x) = \theta e^{-\theta x}\) and consider
the estimator \(\delta_n(x) = \sum_{i = 1}^n\frac{x_i}{n+2}\) of
\(g(\theta) = \frac{1}{theta}.\) Show that for some constants \(c_1\)
and \(c_2\) depending on \(\theta,\)
\[\text{bias of } \delta_n\sim c_1 \text{variance of } \delta_n) \sim \frac{c_2}{n}\]
as \(n\rightarrow \infty.\) The bias of \(\delta_n\) equals its
expectation minus \(\frac{1}{\theta}.\)

Let \(g(\theta) = \frac{1}{\theta}.\) Then
\(f(x) = \frac{1}{g(\theta)} e^{-\frac{x}{g(\theta)}}.\) Therefore,
\(E[x_i] = g(\theta) = \frac{1}{\theta}\) and
\(Var(x_i) = g(\theta)^2 = \frac{1}{\theta^2}\). Consider
\[E[\hat{\delta}_n] = E\left[\sum_{i = 1}^n \frac{x_i}{n+2}\right] = \frac{1}{n+2}\sum_{i=1}^nE[x_i] = \frac{n}{n+2}E[x_i]  = \frac{n}{\theta(n+2)}\]
and
\[Var[\hat{\delta}_n] = Var\left[\sum_{i = 1}^n \frac{x_i}{n+2}\right] = \frac{1}{(n+2)^2}\sum_{i=1}^nVar[x_i]= \frac{n}{(n+2)^2}Var[x_i] = \frac{n}{\theta^2(n+2)^2}.\]
Therefore,
\[Bias(\hat{\delta}_n) = \frac{n}{\theta(n+2)}-\frac{1}{\theta}= \frac{n-n-2}{\theta(n+2)}= -\frac{2}{\theta(n+2)}.\]
Let \(c_1 = -2\theta.\) Then
\[\frac{Bias(\hat{\delta}_n)}{c_1 Var(\hat{\delta}_n)} =\frac{-\frac{2}{\theta(n+2)}}{-2\theta\frac{n}{\theta^2(n_2)^2}}=\frac{-2\theta^2(n+2)^2}{-2\theta^2n(n+2)}= \frac{n^2+4n_4}{n^2+2n}\rightarrow 1.\]
Let \(c_2 = -\frac{2}{\theta}.\) Then
\[\frac{c_1Var(\hat{\delta}_n)}{\frac{c_2}{n}} = \frac{-2\theta\frac{n}{\theta^2(n+2)^2}}{-\frac{2}{\theta}/{n}} = \frac{-2\theta^2n}{-2\theta^2(n+2)^2} = \frac{n^2}{n^2+4n+4} \rightarrow 1.\]
Thus, for \(c_1=-2\theta\) and
\(c_2 = -\frac{2}{\theta},\text{bias of } \delta_n\sim c_1 \text{variance of } \delta_n) \sim \frac{c_2}{n}\).

\hypertarget{exercise-1.24}{%
\subsubsection{Exercise 1.24}\label{exercise-1.24}}

Prove that if \(f(x)\) is everwhere twice differentiable and
\(f''(x)\ge 0\) for \(x\), then \(f(x)\) is convex.

Let \(a = \alpha x+(1-\alpha y).\) Since \(f''(x)\ge 0\), the Mean Value
Theorem implies, \(r_1(x,a)=\frac{1}{2}(x-a)^2f''(x^{*})\ge 0.\),then by
Taylors' expansion, \[\alpha f(x) \ge \alpha f(a) + \alpha(x-a)f'(a)\]
and \[(1-\alpha) f(y) \ge (1-\alpha) f(a) + (1-\alpha)(x-a)f'(a).\]
Therefore, \begin{align*}
\alpha f(x) +(1-\alpha )f(x) & \ge f(a) + f'(a)\left(\alpha(x-a)+(1-\alpha)(y-a)\right)\\
& = f(a)+f'(a)\left(\alpha x - \alpha a + y -\alpha y - a +\alpha a\right)\\
& = f(a) +f'(a)\left([\alpha x + (1-\alpha)y]-a\right)\\
& = f(\alpha x - (1-\alpha)y).
\end{align*} Thus, since
\(\alpha f(x) +(1-\alpha )f(x) \ge f(\alpha x - (1-\alpha)y), f(x)\) is
convex.

\hypertarget{exercise-1.27}{%
\subsubsection{Exercise 1.27}\label{exercise-1.27}}

Recall that \(\log n\) always denotes the natural logarithm of \(n\).
Assuming that \(\log n\) means \(\log_{10} n\) will change some of the
answers in this exercise!

\begin{enumerate}
\def\labelenumi{\alph{enumi}.}
\tightlist
\item
  The following 5 sequences have the property that each tends to 1 as
  \(n\rightarrow\infty\), and for any pair of sequences, one is little-o
  of the other. List them in order of rate of increase from slowest to
  fastest. In other words, give an ordering such that first sequence =
  o(second sequence), second sequence = o(third sequence), etc.
  \begin{align*}
  &n &\sqrt{\log n!} && \sum_{i = 1}^n {^3\sqrt{i}} && 2^{\log n} && (\log n)^{\log\log n}
  \end{align*} Prove the 4 order relationships that result from your
  list.
\end{enumerate}

Hint: Here and in part (b), using a computer to evaluate some of the
sequences for large values of n can be helpful in suggesting the correct
ordering. However, note that this procedure does not constitute a proof!

Ordering the 5 sequences above from slowest to fastest is:
\[(\log n)^{\log\log n}, \sqrt{\log n!}, 2^{\log n}, n,\sum_{i = 1}^n {^3\sqrt{i}}.\]
\newpage

Many of the following proofs depend on the equivalence of a function,
\(f(x)\), existing similar to the sequence on the positive integers and
therefore, uses l'Hopital's rule. Then proving the following:

\begin{enumerate}
\def\labelenumi{(\arabic{enumi})}
\tightlist
\item
  \((\log n)^{\log\log n} =o\left(\sqrt{\log n!}\right)\)
\end{enumerate}

Proceeding by induction. Let n = 6. Then
\(\log n! = \log 6! = 6.57 > 6 = n.\) Assume it is true for \(n = k\)
for \(k\ge6.\) Then let \(n = k+1\). Then
\(\log((k+1)!) = \log((k+1)k!) = \log(k!) + \log(k+1) > k + \log(k+1) > k+1.\)
Thus, for \(n\ge 6\), \(\log n! > n\) implies
\(\sqrt{\log n!} > \sqrt{n}.\) Then since \(-\log\) is a convex
function, taking the \(-\log\) of both sides, we can show that
\(-\log(\log(n))^{\log(\log(n)))} = o\left(-\log{\sqrt{n}}\right).\)
Denote \(a = \log n.\) Consider,

\[\lim_{n\rightarrow\infty} \frac{(\log(\log(n)))^2}{\frac{1}{2}\log n}=\lim_{a\rightarrow \infty}\frac{\log(a)^2}{\frac{1}{2}a}.\]
Then let \(f(x) = \frac{\log(x)^2}{\frac{1}{2}x}\), by l'Hopitals' rule,
taking the derivative two times,\\
\[\lim_{x\rightarrow\infty} \frac{\log(x)^2}{\frac{1}{2}x} = \lim_{x\rightarrow\infty}\frac{2\log a}{a\frac{1}{2}}=\lim_{x\rightarrow\infty}\frac{4}{a^2}=0.\]
Therefore, \(-\log(\log(n))^2 = o\left(-\frac{1}{2}\log n\right)\)
implies \(\log(n)^{\log(\log(n))}=o\left(\sqrt{\log n!}\right)\).

\begin{enumerate}
\def\labelenumi{(\arabic{enumi})}
\setcounter{enumi}{1}
\tightlist
\item
  \(\sqrt{\log n!}=o\left( 2^{\log n}\right)\)
\end{enumerate}

Consider \(\log n! = \log(n) + \log(n+1) + ...\le n\log n.\) and
\(2^{\log n} = e^{\log(2)\log(n)} = n^{\log(2)}.\) Then
\[\lim_{n\rightarrow \infty}\frac{\sqrt{n\log n}}{n^{\log 2}} =\lim_{n\rightarrow \infty}\frac{\sqrt{n}\sqrt{\log n}}{n^{\log 2}}=\lim_{n\rightarrow \infty}\frac{\sqrt{\log n}}{n^{\log 2 - 0.5}}.\]
Consider \(f(x) = \frac{\sqrt{\log x}}{x^{\log 2 - 0.5}}.\) Then by
l'Hopital's rule and order of polynomials,
\[\lim_{x\rightarrow\infty} = \frac{\sqrt{\log x}}{x^{\log 2 - 0.5}} = \lim_{x\rightarrow\infty} = \frac{\frac{1}{2x\sqrt{\log x}}}{\frac{0.19}{x^{0.81}}}=\lim_{x\rightarrow\infty}\frac{x^{0.81}}{(0.19)2x\sqrt{\log x}} = 0.\]
Therefore, \(\frac{\sqrt{n\log n}}{n^{\log 2}}\rightarrow 0\) and
\(\sqrt{n\log n}=o\left( n^{\log 2}\right)\) implies
\(\sqrt{\log n!}=o\left( 2^{\log n}\right)\).

\begin{enumerate}
\def\labelenumi{(\arabic{enumi})}
\setcounter{enumi}{2}
\tightlist
\item
  \(2^{\log n} = o(n)\)
\end{enumerate}

From (2), we know that \(2^{\log n} = n^{\log(2)}.\) Then
\[\frac{n^{\log 2}}{n}=\frac{n^{0.69}}{n^1}\rightarrow 0.\] Therefore,
\(\frac{2^{\log n}}{n}\rightarrow 0\) and \(2^{\log n} = o(n)\).

\begin{enumerate}
\def\labelenumi{(\arabic{enumi})}
\setcounter{enumi}{3}
\tightlist
\item
  \(n = o\left(\sum_{i = 1}^n {^3\sqrt{i}}\right)\)
\end{enumerate}

Using the geometric series, we know
\(\lim_{n\rightarrow \infty}\frac{\frac{4}{3}n^{4/3}}{\sum_{i = 1}^n {^3\sqrt{i}}} \le 1\).
Then,
\[\frac{n}{\sum_{i = 1}^n {^3\sqrt{i}}}=\frac{n}{\frac{4}{3} n^{4/3}}\frac{\frac{4}{3}n^{4/3}}{\sum_{i = 1}^n {^3\sqrt{i}}}\rightarrow 0.\]
Thus, \(n = o\left(\sum_{i = 1}^n {^3\sqrt{i}}\right)\).

\begin{enumerate}
\def\labelenumi{\alph{enumi}.}
\setcounter{enumi}{1}
\tightlist
\item
  Follow the directions of part (a) for the following 13 sequences.
  \begin{align*}
  && \log (n!) && n^2 && n^n && 3^n \\
  &\log(\log n) && n && \log n && 2^{3\log n} && n^{n/2} \\
  && n! && 2^{2^n} && n^{\log n} && (\log n)^n
  \end{align*} Proving the 12 order relationships is challenging but not
  quite as tedious as it sounds; some of the proofs will be very short.
\end{enumerate}

Ordering the 12 sequences above from slowest to fastest is:
\[\log(\log n), \log(n), n, \log(n!), n^2, 2^{3\log n}, n^{\log n}, 3^n, \log(n)^n, n^{n/2}, n!, n^n, 2^{2^n}\]
Then proving the following:

Many of the following proofs depend on the equivalence of a function,
\(f(x)\), existing similar to the sequence on the positive integers and
therefore, uses l'Hopital's rule. Then proving the following:

\begin{enumerate}
\def\labelenumi{(\arabic{enumi})}
\tightlist
\item
  \(\log(\log n) = o\left(\log(n)\right)\)
\end{enumerate}

Consider \(f(x) = \frac{\log\log n}{\log n}.\) Then by l'Hopital's rule,
\[\lim_{x\rightarrow\infty}\frac{\log\log x}{\log x} = \lim_{x\rightarrow\infty}\frac{\frac{1}{x\log x}}{\frac{1}{x}}=\lim_{x\rightarrow\infty}=\frac{1}{\log x}=0.\]
Therefore, \(\frac{\log\log n}{\log n}\rightarrow 0.\)

\begin{enumerate}
\def\labelenumi{(\arabic{enumi})}
\setcounter{enumi}{1}
\tightlist
\item
  \(\log(n)= o\left( n \right)\)
\end{enumerate}

Consdier \(f(x) = \frac{\log x}{x}.\) Then by l'Hopital's rule,
\[\lim_{x\rightarrow\infty}\frac{\log x}{x} = \lim_{x\rightarrow\infty}\frac{\frac{1}{x}}{x} = \lim_{x\rightarrow\infty}\frac{1}{x^2} = 0.\]
Therefore, \(\frac{\log n}{n}\rightarrow 0.\)

\begin{enumerate}
\def\labelenumi{(\arabic{enumi})}
\setcounter{enumi}{2}
\tightlist
\item
  \(n = o\left( \log(n!)\right)\)
\end{enumerate}

In part a (1) above, we used induction to show that \(n>\log n!\) for
\(n\ge 6\). Thus, \(n = o\left( \log(n!)\right)\).

\begin{enumerate}
\def\labelenumi{(\arabic{enumi})}
\setcounter{enumi}{3}
\tightlist
\item
  \(\log(n!)= o\left( n^2 \right)\)
\end{enumerate}

Recall, \(\log n! \le n \log n.\) Then using (2) from above,
\[\lim_{n\rightarrow\infty} \frac{n\log n}{n^2} = \lim_{n\rightarrow\infty}\frac{\log n}{n} =0.\]
Therefore, \(\frac{\log n!}{n^2}\rightarrow 0.\)

\begin{enumerate}
\def\labelenumi{(\arabic{enumi})}
\setcounter{enumi}{4}
\tightlist
\item
  \(n^2= o\left( 2^{3\log n}\right)\)
\end{enumerate}

Recall from part (a), \(2^{\log n} = n^{\log 2}.\) Then
\[\lim_{n \rightarrow \infty} \frac{n^2}{2^{3\log n}} = \lim_{n \rightarrow \infty} \frac{n^2}{n^{3\log 2}}=\lim_{n \rightarrow \infty} \frac{n^2}{n^{2.08}}=0.\]
Therefore, \(\frac{n^2}{2^{3\log n}}\rightarrow 0\).

\begin{enumerate}
\def\labelenumi{(\arabic{enumi})}
\setcounter{enumi}{5}
\tightlist
\item
  \(2^{3\log n}= o\left( n^{\log n}\right)\)
\end{enumerate}

Notice, \(\log n > 3\log 2\) for \(n>8\). Then,
\[\lim_{n\rightarrow \infty}\frac{n^{3 \log 2}}{n^{\log n}}=0.\]
Therefore, \(\frac{n^{3 \log 2}}{n^{\log n}}\rightarrow 0\).

\begin{enumerate}
\def\labelenumi{(\arabic{enumi})}
\setcounter{enumi}{6}
\tightlist
\item
  \(n^{\log n}= o\left(3^n\right)\)
\end{enumerate}

Consider the convex function, \(f(x) = e^{\log x}.\) Then, from equation
1.26, with \(\alpha = 2\) and \(\beta = 1,\)

\[\lim_{n\rightarrow\infty}\frac{n^{\log n}}{3^n}=\lim_{n\rightarrow\infty}\frac{e^{\log(n)^2}}{e^{n\log(3)}}=0\]
since logarithm gorws slower than polynomial. Therefore,
\(n^{\log n}= o\left(3^n\right)\).

\begin{enumerate}
\def\labelenumi{(\arabic{enumi})}
\setcounter{enumi}{7}
\tightlist
\item
  \(3^n= o\left( \log(n)^n\right)\)
\end{enumerate}

Consider the convex function, \(f(x) = e^{\log x}.\) Then,
\[\lim_{n\rightarrow\infty}\frac{3^n}{\log(n)^n}=\lim_{n\rightarrow\infty}\frac{e^{n\log 3}}{e^{n\log n}}=\lim_{n\rightarrow\infty}\frac{(e^n)^{\log 3}}{(e^n)^{\log n}}=0.\]
Therefore, \(3^n= o\left( \log(n)^n\right)\).

\begin{enumerate}
\def\labelenumi{(\arabic{enumi})}
\setcounter{enumi}{8}
\tightlist
\item
  \(\log(n)^n= o\left(n^{n/2}\right)\)
\end{enumerate}

Consider \(f(x) = \frac{\log x}{x^{1/2}}.\) Then by l'Hopitals 's rule,
taking the derivative twice,
\[\lim_{x\rightarrow\infty}\frac{\log x}{x^{1/2}}=\lim_{x\rightarrow\infty}\frac{1/x}{\frac{1}{2}x^{-1/2}}=\lim_{x\rightarrow\infty}\frac{2x^{1/2}}{x}=0.\]
Therefore, \(\frac{\log(n)^n}{(n^{1/2})^n}\rightarrow 0\) and
\(\log(n)^n= o\left(n^{n/2}\right)\).

\begin{enumerate}
\def\labelenumi{(\arabic{enumi})}
\setcounter{enumi}{9}
\tightlist
\item
  \(n^{n/2}= o\left(n!\right)\)
\end{enumerate}

Consider
\[\lim_{n\rightarrow\infty} \frac{n^{n/2}}{n!}=\lim_{n\rightarrow\infty} \frac{n^{n/2}}{n^n + ...}=0.\]
Therefore, \(n^{n/2} = o\left( n!\right)\).

\begin{enumerate}
\def\labelenumi{(\arabic{enumi})}
\setcounter{enumi}{10}
\tightlist
\item
  \(n!= o\left(n^n\right)\)
\end{enumerate}

Let \(n = 2.\) Then \(2! = 2 < 4 = 2^2\). Assume it is true for
\(n =k.\) Let \(n = k+1.\) Then
\[(k+1)! = (k+1)k!<(k+1)k^k<(k+1)(k+1)^k=(k+1)^{k+1}.\] Therefore, for
all \(n\ge 2, n! < n^n\). Thus, \(n!= o\left(n^n\right).\)

\begin{enumerate}
\def\labelenumi{(\arabic{enumi})}
\setcounter{enumi}{11}
\tightlist
\item
  \(n^n = o\left(n^{2^n}\right)\)
\end{enumerate}

Consider \(f(x) = \frac{x\log x}{2^x}.\) Then using l'Hopital's rule,
\[\lim_{x\rightarrow\infty}\frac{x\log x}{2^x}=\lim_{x\rightarrow\infty}\frac{1+\log n}{2^n\log(2)}=\lim_{x\rightarrow\infty}\frac{1}{n(2^n\log(2)^2)}=0.\]
Thus \(\frac{n\log n}{2^n}\rightarrow 0\). Then taking the convex
function, \(f(x) = e^{\log(x)}\),
\[\lim_{n\rightarrow\infty}\frac{n^n}{2^{2^n}}=\lim_{n\rightarrow\infty}\frac{e^{n\log n}}{e^{2^n\log2}}=\lim_{n\rightarrow\infty}e^{n\log n-2^n\log 2}=\lim_{n\rightarrow\infty}e^{2^n(\frac{n\log n}{2^n}-\log 2)}=\lim_{n\rightarrow\infty}e^{-2^n\log 2}=0.\]
Therefore, \(n^n = o\left(n^{2^n}\right)\).

\hypertarget{exercise-1.29}{%
\subsubsection{Exercise 1.29}\label{exercise-1.29}}

Suppose that \(a_{nj} \rightarrow c_j\) as \(n\rightarrow \infty\) for
\(j = 1, ..., k.\) Prove that if
\(f:\mathbb{R}^k \rightarrow \mathbb{R}\) is continuous at the point
\(\boldsymbol{c}\), then
\(f(\boldsymbol{a}_n)\rightarrow f(\boldsymbol{c}).\) This proves every
part of Exercise 1.1. (The hard work of an exercise like 1.1(b) is in
showing that multiplication is continuous.)

\begin{proof}
We need to show that for any $\epsilon>0,$ there exists an $N$ such that for all $n>n$, $||f(\boldsymbol{a_n})-f(\boldsymbol{c})||<\epsilon$. Then from the definition of continuity, we know there exists some $\delta >0$ such that $||f(\boldsymbol{x})-f(\boldsymbol{c})||<\epsilon$ for all $x$ such that $||\boldsymbol{x}-\boldsymbol{c}||<\delta.$ Since $a_{nj}\rightarrow c_j$ for all $1\le j \le k$, then we know $\boldsymbol{a_n}\rightarrow \boldsymbol{c}$ as $n\rightarrow \infty$. Then since $\delta >0,$ there must by definition be some $N$ such that $||\boldsymbol{a_n}-\boldsymbol{c}||<\delta$ for all $n>N.$ We conclude that for all $n>N$, $||f(\boldsymbol{a_n})-f(\boldsymbol{c})||<\epsilon.$
\end{proof}

\hypertarget{exercise-1.31}{%
\subsubsection{Exercise 1.31}\label{exercise-1.31}}

Prove that the converse of Theorem 1.38 is not true by finding a
function that is not differentiable at some point but whose partial
derivatives at that point all exist.

Consider \(f(x,y) = I(xy = 0).\) Then along the \(x\) and \(y\) axis,
\(f(x,y) = 1.\) Therefore, the partial derivatives with respect to both
\(x\) and \(y\) exist at the orgin where the two lines cross. However,
since \(f(x,y) = 0\) everywhere else, \(f(x,y)\) is not continuous at
the orgin. Thus, \(\triangledown f(x,y)\) does not exist.

\hypertarget{exercise-1.34}{%
\subsubsection{Exercise 1.34}\label{exercise-1.34}}

\begin{enumerate}
\def\labelenumi{\alph{enumi}.}
\tightlist
\item
  Find the Hessian matrix of the loglikelihood function defined in
  Exercise 1.32.
\end{enumerate}

Consider \begin{align*}
&&f(\boldsymbol{x};\mu, \sigma^2) & = \frac{exp\{-\frac{1}{2\sigma^2}(x_i-\mu)^2\}}{\sqrt(2\pi\sigma^2)}\\
&\implies& L(\mu,\sigma^2;\boldsymbol{x}) &= \prod_{i = 1}^n \frac{1}{\sqrt(2\pi\sigma^2)}e^{-\frac{1}{2\sigma^2}(x_i-\mu)^2}\\
&&&=(s\pi\sigma^2)^{-n/2}e^{-\frac{1}{2\sigma^2}\sum_{i=1}^n(x_i-\mu^2)}\\
&\implies& \log L(\mu,\sigma^2;\boldsymbol{x}) &= -\frac{n}{2}\log(2\pi\sigma^2) - \frac{1}{2\sigma^2}\sum_{i = 1}^n (x_i-\mu)^2.
\end{align*}

Then taking partial derivatives,
\[\triangledown \log L(\mu,\sigma^2;\boldsymbol{x}) = 
\begin{bmatrix}
\frac{d^2}{d\mu^2}\log L(\mu,\sigma^2;\boldsymbol{x}) & \frac{d^2}{d\mu d\sigma^2}\log L(\mu,\sigma^2;\boldsymbol{x}) \\
\frac{d^2}{d\mu d\sigma^2}\log L(\mu,\sigma^2;\boldsymbol{x}) & \frac{d^2}{d\sigma^4}\log L(\mu,\sigma^2;\boldsymbol{x})\\
\end{bmatrix} =
\begin{bmatrix}
-\frac{n}{\sigma^2} & -\frac{\sum_{i=1}^n(x_i-\mu)}{\sigma^4}\\
-\frac{\sum_{i=1}^n(x_i-\mu)}{\sigma^4} & \frac{n}{2\sigma^4}-\frac{\sum_{i=1}^n(x_i-\mu)^2}{\sigma^6}\\
\end{bmatrix} \]

\begin{enumerate}
\def\labelenumi{\alph{enumi}.}
\setcounter{enumi}{1}
\tightlist
\item
  Suppose that \(n = 10\) and that we observe this sample:
\end{enumerate}

\begin{Shaded}
\begin{Highlighting}[]
\NormalTok{x <-}\StringTok{ }\KeywordTok{c}\NormalTok{(}\FloatTok{2.946}\NormalTok{, }\FloatTok{0.975}\NormalTok{, }\FloatTok{1.333}\NormalTok{, }\FloatTok{4.484}\NormalTok{, }\FloatTok{1.711}\NormalTok{, }\FloatTok{2.627}\NormalTok{, }\FloatTok{-0.628}\NormalTok{, }
    \FloatTok{2.476}\NormalTok{, }\FloatTok{2.599}\NormalTok{, }\FloatTok{2.143}\NormalTok{)}
\KeywordTok{kable}\NormalTok{(}\KeywordTok{t}\NormalTok{(x))}
\end{Highlighting}
\end{Shaded}

\begin{longtable}[]{@{}rrrrrrrrrr@{}}
\toprule
\endhead
2.946 & 0.975 & 1.333 & 4.484 & 1.711 & 2.627 & -0.628 & 2.476 & 2.599 &
2.143\tabularnewline
\bottomrule
\end{longtable}

Evaluate the Hessian matrix at the maximum likelihood estimator
\((\hat{\mu}, \hat{\sigma}^2)\). (A formula for the MLE is given in the
hint to Exercise 1.32)

\begin{Shaded}
\begin{Highlighting}[]
\NormalTok{n <-}\StringTok{ }\DecValTok{10}
\NormalTok{muhat <-}\StringTok{ }\KeywordTok{mean}\NormalTok{(x)}
\NormalTok{sigma2hat <-}\StringTok{ }\KeywordTok{mean}\NormalTok{((x }\OperatorTok{-}\StringTok{ }\NormalTok{muhat)}\OperatorTok{^}\DecValTok{2}\NormalTok{)}
\NormalTok{Hessian <-}\StringTok{ }\KeywordTok{matrix}\NormalTok{(}\KeywordTok{c}\NormalTok{(}\OperatorTok{-}\NormalTok{n}\OperatorTok{/}\NormalTok{sigma2hat, }\OperatorTok{-}\NormalTok{(}\KeywordTok{sum}\NormalTok{(x }\OperatorTok{-}\StringTok{ }\KeywordTok{rep}\NormalTok{(muhat, }
\NormalTok{    n)))}\OperatorTok{/}\NormalTok{sigma2hat}\OperatorTok{^}\DecValTok{2}\NormalTok{, }\OperatorTok{-}\NormalTok{(}\KeywordTok{sum}\NormalTok{(x }\OperatorTok{-}\StringTok{ }\KeywordTok{rep}\NormalTok{(muhat, n)))}\OperatorTok{/}\NormalTok{sigma2hat}\OperatorTok{^}\DecValTok{2}\NormalTok{, }
\NormalTok{    n}\OperatorTok{/}\NormalTok{(}\DecValTok{2} \OperatorTok{*}\StringTok{ }\NormalTok{sigma2hat}\OperatorTok{^}\DecValTok{2}\NormalTok{) }\OperatorTok{-}\StringTok{ }\NormalTok{(}\KeywordTok{sum}\NormalTok{((x }\OperatorTok{-}\StringTok{ }\KeywordTok{rep}\NormalTok{(muhat, n))}\OperatorTok{^}\DecValTok{2}\NormalTok{))}\OperatorTok{/}\NormalTok{sigma2hat}\OperatorTok{^}\DecValTok{3}\NormalTok{), }
    \DecValTok{2}\NormalTok{, }\DecValTok{2}\NormalTok{, }\DataTypeTok{byrow =}\NormalTok{ T)}
\KeywordTok{kable}\NormalTok{(Hessian)}
\end{Highlighting}
\end{Shaded}

\begin{longtable}[]{@{}rr@{}}
\toprule
\endhead
-6.0587 & 0.000000\tabularnewline
0.0000 & -1.835392\tabularnewline
\bottomrule
\end{longtable}


\end{document}
