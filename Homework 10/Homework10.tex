\documentclass[12pt,]{article}
\usepackage{lmodern}
\usepackage{amssymb,amsmath}
\usepackage{ifxetex,ifluatex}
\usepackage{fixltx2e} % provides \textsubscript
\ifnum 0\ifxetex 1\fi\ifluatex 1\fi=0 % if pdftex
  \usepackage[T1]{fontenc}
  \usepackage[utf8]{inputenc}
\else % if luatex or xelatex
  \ifxetex
    \usepackage{mathspec}
  \else
    \usepackage{fontspec}
  \fi
  \defaultfontfeatures{Ligatures=TeX,Scale=MatchLowercase}
\fi
% use upquote if available, for straight quotes in verbatim environments
\IfFileExists{upquote.sty}{\usepackage{upquote}}{}
% use microtype if available
\IfFileExists{microtype.sty}{%
\usepackage{microtype}
\UseMicrotypeSet[protrusion]{basicmath} % disable protrusion for tt fonts
}{}
\usepackage[margin=1in]{geometry}
\usepackage{hyperref}
\hypersetup{unicode=true,
            pdftitle={Homework 10},
            pdfauthor={Emily Robinson},
            pdfborder={0 0 0},
            breaklinks=true}
\urlstyle{same}  % don't use monospace font for urls
\usepackage{graphicx,grffile}
\makeatletter
\def\maxwidth{\ifdim\Gin@nat@width>\linewidth\linewidth\else\Gin@nat@width\fi}
\def\maxheight{\ifdim\Gin@nat@height>\textheight\textheight\else\Gin@nat@height\fi}
\makeatother
% Scale images if necessary, so that they will not overflow the page
% margins by default, and it is still possible to overwrite the defaults
% using explicit options in \includegraphics[width, height, ...]{}
\setkeys{Gin}{width=\maxwidth,height=\maxheight,keepaspectratio}
\IfFileExists{parskip.sty}{%
\usepackage{parskip}
}{% else
\setlength{\parindent}{0pt}
\setlength{\parskip}{6pt plus 2pt minus 1pt}
}
\setlength{\emergencystretch}{3em}  % prevent overfull lines
\providecommand{\tightlist}{%
  \setlength{\itemsep}{0pt}\setlength{\parskip}{0pt}}
\setcounter{secnumdepth}{0}
% Redefines (sub)paragraphs to behave more like sections
\ifx\paragraph\undefined\else
\let\oldparagraph\paragraph
\renewcommand{\paragraph}[1]{\oldparagraph{#1}\mbox{}}
\fi
\ifx\subparagraph\undefined\else
\let\oldsubparagraph\subparagraph
\renewcommand{\subparagraph}[1]{\oldsubparagraph{#1}\mbox{}}
\fi

%%% Use protect on footnotes to avoid problems with footnotes in titles
\let\rmarkdownfootnote\footnote%
\def\footnote{\protect\rmarkdownfootnote}

%%% Change title format to be more compact
\usepackage{titling}

% Create subtitle command for use in maketitle
\providecommand{\subtitle}[1]{
  \posttitle{
    \begin{center}\large#1\end{center}
    }
}

\setlength{\droptitle}{-2em}

  \title{Homework 10}
    \pretitle{\vspace{\droptitle}\centering\huge}
  \posttitle{\par}
  \subtitle{STAT 984}
  \author{Emily Robinson}
    \preauthor{\centering\large\emph}
  \postauthor{\par}
      \predate{\centering\large\emph}
  \postdate{\par}
    \date{December 5, 2019}

\usepackage{amsmath}
\usepackage{amssymb}
\usepackage{amsthm}

\begin{document}
\maketitle

\hypertarget{exercise-8.1}{%
\subsubsection{Exercise 8.1}\label{exercise-8.1}}

Let \(X_1,...,X_n\) be a simple random sample from a Pareto distribution
with density \[f(x)=\theta c^\theta x^{-(\theta+1)}I\{x>c\}\] for a
known constant \(c>0\) and parameter \(\theta>0.\) Derive the Wald, Rao,
and likelihood ratio tests of \(\theta=\theta_0\) against a two-sided
alternative.

\hypertarget{exercise-8.2}{%
\subsubsection{Exercise 8.2}\label{exercise-8.2}}

Suppose that \(\boldsymbol{X}\) is multinomial\((n,\boldsymbol{p})\),
where \(\boldsymbol{p}\in\mathbb{R}^k\). In order to satisfy the
regularity condition that the parameter space be an open set, define
\(\boldsymbol\theta=(p_1,...,p_{k-1})\). Suppose that we wish to test
\(H_0:\boldsymbol\theta=\boldsymbol\theta^0\) against
\(H_1:\boldsymbol\theta\ne\boldsymbol\theta^0.\)

\begin{enumerate}
\def\labelenumi{(\alph{enumi})}
\item
  Prove that the Wald and score tests are the same as the usual Pearson
  chi-square test.
\item
  Derive the likelihood ratio statistic \(2\Delta_n\).
\end{enumerate}

\hypertarget{exercise-8.8}{%
\subsubsection{Exercise 8.8}\label{exercise-8.8}}

Let \(X_1,...,X_n\) be an independent sample from an exponential
distribution with mean \(\lambda,\) and \(Y_1,...,Y_n\) be an
independent sample from an exponential distribution with mean \(\mu.\)
Assume that \(X_i\) and \(Y_i\) are independent. We are interested in
testing the hypothesis \(H_0: \lambda = \mu\) verses
\(H_1: \lambda > \mu\). Consider the statistic
\[T_n=2\sum_{i=1}^n(I_i-1/2)/\sqrt{n},\] where \(I_i\) is the indicator
variable \(I_i=I(X_i>Y_i).\)

\begin{enumerate}
\def\labelenumi{(\alph{enumi})}
\item
  Derive the asymptotic distribution of \(T_n\) under the null
  hypothesis.
\item
  Use the Lindeberg Theorem to show that, under the local alternative
  hypothesis \((\lambda_n,\mu_n)=)\lambda+n^{-1/2}\delta,\lambda),\)
  where \(\delta>0\),
  \[\frac{\sum_{i=1}^n(I_i-\rho_n)}{\sqrt{n\rho_n(1-\rho_n)}}\overset{\mathbb{L}}\rightarrow N(0,1), \text{ where }\rho_n=\frac{\lambda_n}{\lambda_n+\mu_n}=\frac{\lambda+n^{-1/2}\delta}{2\lambda+n^{-1/2}\lambda}.\]
\end{enumerate}

\hypertarget{exercise-8.9}{%
\subsubsection{Exercise 8.9}\label{exercise-8.9}}

Suppose \(X_1,...X_m\) is a simple random sample and \(Y_1,...,Y_n\) is
another simple random sample independent of the \(X_i\), with
\(P(X_i\le t)=t^2\) for \(t\in [0,1]\) and \(P(Y_i\le t)=(t-\theta)^2\)
for \(t\in [\theta, \theta+1]\). Assume \(m/(m+n)\rightarrow \rho\) as
\(m,n\rightarrow \infty\) and \(0<\theta<1.\)

Find the asymptotic distribution of
\(\sqrt{m+n}[g(\bar Y-\bar X)-g(\theta)].\)


\end{document}
