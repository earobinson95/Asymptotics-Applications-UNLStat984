\documentclass[12pt,]{article}
\usepackage{lmodern}
\usepackage{amssymb,amsmath}
\usepackage{ifxetex,ifluatex}
\usepackage{fixltx2e} % provides \textsubscript
\ifnum 0\ifxetex 1\fi\ifluatex 1\fi=0 % if pdftex
  \usepackage[T1]{fontenc}
  \usepackage[utf8]{inputenc}
\else % if luatex or xelatex
  \ifxetex
    \usepackage{mathspec}
  \else
    \usepackage{fontspec}
  \fi
  \defaultfontfeatures{Ligatures=TeX,Scale=MatchLowercase}
\fi
% use upquote if available, for straight quotes in verbatim environments
\IfFileExists{upquote.sty}{\usepackage{upquote}}{}
% use microtype if available
\IfFileExists{microtype.sty}{%
\usepackage{microtype}
\UseMicrotypeSet[protrusion]{basicmath} % disable protrusion for tt fonts
}{}
\usepackage[margin=1in]{geometry}
\usepackage{hyperref}
\hypersetup{unicode=true,
            pdftitle={Homework 7},
            pdfauthor={Emily Robinson},
            pdfborder={0 0 0},
            breaklinks=true}
\urlstyle{same}  % don't use monospace font for urls
\usepackage{color}
\usepackage{fancyvrb}
\newcommand{\VerbBar}{|}
\newcommand{\VERB}{\Verb[commandchars=\\\{\}]}
\DefineVerbatimEnvironment{Highlighting}{Verbatim}{commandchars=\\\{\}}
% Add ',fontsize=\small' for more characters per line
\usepackage{framed}
\definecolor{shadecolor}{RGB}{248,248,248}
\newenvironment{Shaded}{\begin{snugshade}}{\end{snugshade}}
\newcommand{\AlertTok}[1]{\textcolor[rgb]{0.94,0.16,0.16}{#1}}
\newcommand{\AnnotationTok}[1]{\textcolor[rgb]{0.56,0.35,0.01}{\textbf{\textit{#1}}}}
\newcommand{\AttributeTok}[1]{\textcolor[rgb]{0.77,0.63,0.00}{#1}}
\newcommand{\BaseNTok}[1]{\textcolor[rgb]{0.00,0.00,0.81}{#1}}
\newcommand{\BuiltInTok}[1]{#1}
\newcommand{\CharTok}[1]{\textcolor[rgb]{0.31,0.60,0.02}{#1}}
\newcommand{\CommentTok}[1]{\textcolor[rgb]{0.56,0.35,0.01}{\textit{#1}}}
\newcommand{\CommentVarTok}[1]{\textcolor[rgb]{0.56,0.35,0.01}{\textbf{\textit{#1}}}}
\newcommand{\ConstantTok}[1]{\textcolor[rgb]{0.00,0.00,0.00}{#1}}
\newcommand{\ControlFlowTok}[1]{\textcolor[rgb]{0.13,0.29,0.53}{\textbf{#1}}}
\newcommand{\DataTypeTok}[1]{\textcolor[rgb]{0.13,0.29,0.53}{#1}}
\newcommand{\DecValTok}[1]{\textcolor[rgb]{0.00,0.00,0.81}{#1}}
\newcommand{\DocumentationTok}[1]{\textcolor[rgb]{0.56,0.35,0.01}{\textbf{\textit{#1}}}}
\newcommand{\ErrorTok}[1]{\textcolor[rgb]{0.64,0.00,0.00}{\textbf{#1}}}
\newcommand{\ExtensionTok}[1]{#1}
\newcommand{\FloatTok}[1]{\textcolor[rgb]{0.00,0.00,0.81}{#1}}
\newcommand{\FunctionTok}[1]{\textcolor[rgb]{0.00,0.00,0.00}{#1}}
\newcommand{\ImportTok}[1]{#1}
\newcommand{\InformationTok}[1]{\textcolor[rgb]{0.56,0.35,0.01}{\textbf{\textit{#1}}}}
\newcommand{\KeywordTok}[1]{\textcolor[rgb]{0.13,0.29,0.53}{\textbf{#1}}}
\newcommand{\NormalTok}[1]{#1}
\newcommand{\OperatorTok}[1]{\textcolor[rgb]{0.81,0.36,0.00}{\textbf{#1}}}
\newcommand{\OtherTok}[1]{\textcolor[rgb]{0.56,0.35,0.01}{#1}}
\newcommand{\PreprocessorTok}[1]{\textcolor[rgb]{0.56,0.35,0.01}{\textit{#1}}}
\newcommand{\RegionMarkerTok}[1]{#1}
\newcommand{\SpecialCharTok}[1]{\textcolor[rgb]{0.00,0.00,0.00}{#1}}
\newcommand{\SpecialStringTok}[1]{\textcolor[rgb]{0.31,0.60,0.02}{#1}}
\newcommand{\StringTok}[1]{\textcolor[rgb]{0.31,0.60,0.02}{#1}}
\newcommand{\VariableTok}[1]{\textcolor[rgb]{0.00,0.00,0.00}{#1}}
\newcommand{\VerbatimStringTok}[1]{\textcolor[rgb]{0.31,0.60,0.02}{#1}}
\newcommand{\WarningTok}[1]{\textcolor[rgb]{0.56,0.35,0.01}{\textbf{\textit{#1}}}}
\usepackage{longtable,booktabs}
\usepackage{graphicx,grffile}
\makeatletter
\def\maxwidth{\ifdim\Gin@nat@width>\linewidth\linewidth\else\Gin@nat@width\fi}
\def\maxheight{\ifdim\Gin@nat@height>\textheight\textheight\else\Gin@nat@height\fi}
\makeatother
% Scale images if necessary, so that they will not overflow the page
% margins by default, and it is still possible to overwrite the defaults
% using explicit options in \includegraphics[width, height, ...]{}
\setkeys{Gin}{width=\maxwidth,height=\maxheight,keepaspectratio}
\IfFileExists{parskip.sty}{%
\usepackage{parskip}
}{% else
\setlength{\parindent}{0pt}
\setlength{\parskip}{6pt plus 2pt minus 1pt}
}
\setlength{\emergencystretch}{3em}  % prevent overfull lines
\providecommand{\tightlist}{%
  \setlength{\itemsep}{0pt}\setlength{\parskip}{0pt}}
\setcounter{secnumdepth}{0}
% Redefines (sub)paragraphs to behave more like sections
\ifx\paragraph\undefined\else
\let\oldparagraph\paragraph
\renewcommand{\paragraph}[1]{\oldparagraph{#1}\mbox{}}
\fi
\ifx\subparagraph\undefined\else
\let\oldsubparagraph\subparagraph
\renewcommand{\subparagraph}[1]{\oldsubparagraph{#1}\mbox{}}
\fi

%%% Use protect on footnotes to avoid problems with footnotes in titles
\let\rmarkdownfootnote\footnote%
\def\footnote{\protect\rmarkdownfootnote}

%%% Change title format to be more compact
\usepackage{titling}

% Create subtitle command for use in maketitle
\providecommand{\subtitle}[1]{
  \posttitle{
    \begin{center}\large#1\end{center}
    }
}

\setlength{\droptitle}{-2em}

  \title{Homework 7}
    \pretitle{\vspace{\droptitle}\centering\huge}
  \posttitle{\par}
  \subtitle{STAT 984}
  \author{Emily Robinson}
    \preauthor{\centering\large\emph}
  \postauthor{\par}
      \predate{\centering\large\emph}
  \postdate{\par}
    \date{November 7, 2019}

\usepackage{amsmath}
\usepackage{amssymb}
\usepackage{amsthm}

\begin{document}
\maketitle

\hypertarget{exercise-5.3}{%
\subsubsection{Exercise 5.3}\label{exercise-5.3}}

Suppose \(X_n \sim \text{binomial}(n,p)\), where \(0<p<1\).

\begin{enumerate}
\def\labelenumi{(\alph{enumi})}
\tightlist
\item
  Find the asymptotic distribution of \(g(X_n/n)-g(p)\), where
  \(g(x) = \min\{x,1-x\}.\)
\end{enumerate}

By the CLT, we know
\[\sqrt{n}\left(\frac{X_n}{n}-p\right)\overset{d}\rightarrow N\left(0, p(1-p)\right).\]
Then \[g(x) = \begin{cases}
                                   p & \text{when } p\in (0,1/2) \\
                                   1-p & \text{when } p\in (1/2,1).
        \end{cases}\] Therefore, \(g'(p)=1\implies [g''(p)]^2=1\) when
\(p \in \{(0,1/2)\cup (1/2,1)\}.\) Then by the delta method,
\[\sqrt{n}\left(g\left(\frac{X_n}{n}\right)-g(p)\right)\overset{d}\rightarrow N(0,p(1-p).\]
However, consider \(p=1/2.\) Then \(g'(p)\) does not exist, thus, the
delta method does not apply. Therefore, note
\[\sqrt{n}\left(g\left(\frac{X_n}{n}\right)-g(1/2)\right)=-\sqrt{n}\left|\frac{X_n}{n}-1/2\right|.\]
Thus, since the absolute value is a continuous function, by the CLT, for
\(p=1/2\), we obtain
\[\sqrt{n}\left(g\left(\frac{X_n}{n}\right)-g(p)\right)\overset{d}\rightarrow -\sqrt{p(1-p)}|Z|\]
where \(Z\sim N(0,1)\).

\begin{enumerate}
\def\labelenumi{(\alph{enumi})}
\setcounter{enumi}{1}
\item
  Show that \(h(x) = \sin ^{-1}(\sqrt{x})\) is a variance-stabilizing
  transformation for \(X_n/n\). This is called the
  \textit{arcsine transformation} of a sample proportion.

  \textbf{Hint:} \((d/du)\text{sin}^{-1}(u) = 1/\sqrt{1-u^2}\).
\end{enumerate}

Consider \(h(x)=\sin^{-1}(\sqrt{x}).\) Then by the chain rule,
\(h'(x)=\frac{1}{\sqrt{1-x}}\frac{1}{2\sqrt{x}}=\frac{1}{2\sqrt{x(1-x)}}\).
Then by the CLT and delta method,
\[\sqrt{n}\left[h\left(\frac{X_n}{n}\right)-h(p)\right]\overset{d}\rightarrow N\left(0,[h'(p)]^2p(1-p)\right)=N\left(0,\frac{p(1-p)}{4p(1-p)}\right)=N\left(0,\frac{1}{4}\right).\]

\hypertarget{exercise-5.4}{%
\subsubsection{Exercise 5.4}\label{exercise-5.4}}

Let \(X_1, X_2, ...\) be independent from \(N(\mu,\sigma^2)\) where
\(\mu \ne 0\). Let \[s_n^2=\frac{1}{n}\sum_{i=1}^n(X_i-\bar X_n)^2.\]
Find the asymptotic distribution of the coefficient of variation
\(S_n/\bar X_n.\)

Note \(s_n\overset{p}\rightarrow \sigma\) and
\(\bar X_n \overset{P}\rightarrow \mu\) implies
\(s_n/\bar X_n\overset{P}\rightarrow \sigma/\mu.\) Therefore,
\(s_n/\bar X_n\) is a consistent estimator for \(\sigma/\mu.\) Then
\[\sqrt{n}\left(s_n/\bar X_n-\sigma/\mu\right)=\frac{1}{\bar X_n}\sqrt{n}\left(s_n-\frac{\sigma}{\mu}\bar X_n\right)=\frac{1}{\bar X_n}\left(\sqrt{n}(s_n-\sigma)-\sqrt{n}\left(\frac{\sigma}{\mu}\bar X_n\right)\right).\]
Let \(g(x) = \frac{\sigma}{\mu}x.\) Then \(g'(x)=\frac{\sigma}{\mu}.\)
Therefore, by the CLT and delta method, \begin{align*}
&&\sqrt{n}(\bar X_n-\mu)&\overset{d}\rightarrow N(0,\sigma^2)\\
&\implies&\sqrt{n}(g(\bar X_n)-g(\mu))&\overset{d}\rightarrow N(0,[g'(\mu)]^2\sigma^2)\\
&\implies&\sqrt{n}(\frac{\sigma}{\mu}\bar X_n-\sigma)&\overset{d}\rightarrow N(0,\sigma^4/\mu^2).\\
\end{align*} Then consider
\[\sqrt{n}(s_n-\sigma)=\sqrt{n}\left(\frac{s_n^2-\sigma^2}{s_n+\sigma}\right).\]
Note, \(s_n+\sigma\overset{P}\rightarrow 2\sigma\) and
\(\sqrt{n}(s_n^2-\sigma^2)\overset{d}\rightarrow N(0,2\sigma^4).\) Then
by Slutsky's Theorem,
\[\sqrt{n}\left(\frac{s_n^2-\sigma^2}{s_n+\sigma}\right)\overset{d}\rightarrow \frac{1}{2\sigma}N(0,2\sigma^4)=N\left(0, \frac{2\sigma^4}{(2\sigma)^2}\right)=N\left(0, \frac{\sigma^2}{2}\right)\].
Then, since \(\bar X_n\) and \(s_n\) are independent,
\[\sqrt{n}(s_n-\sigma)-\sqrt{n}\left(\frac{\sigma}{\mu}\bar X_n\right)\overset{d}\rightarrow N\left(0, \frac{\sigma^4}{\mu^2}+\frac{\sigma^2}{2}\right).\]
Recall, \(\bar X_n \overset{P}\rightarrow \mu.\) Then by Slutksy's
Theorem,
\[\frac{1}{\bar X_n}\left(\sqrt{n}(s_n-\sigma)-\sqrt{n}\left(\frac{\sigma}{\mu}\bar X_n\right)\right)\overset{d}\rightarrow \frac{1}{\mu}N\left(0, \frac{\sigma^4}{\mu^2}+\frac{\sigma^2}{2}\right)=N\left(0, \frac{\sigma^4}{\mu^4}+\frac{\sigma^2}{2\mu^2}\right).\]

\hypertarget{exercise-5.5}{%
\subsubsection{Exercise 5.5}\label{exercise-5.5}}

Let \(X_n\sim \text{binomial}(n,p)\), where \(p\in(0,1)\) is unknown.
Obtain confidence intervals for \(p\) in two different ways:

\begin{enumerate}
\def\labelenumi{(\alph{enumi})}
\tightlist
\item
  Since \(\sqrt{n}(X_n/n-p)\overset{d}\rightarrow N[0,p(1-p)]\), the
  variance of the limiting distribution depends only on \(p\). Use the
  fact that \(X_n/n\overset{P}\rightarrow p\) to find a consistent
  estimator of the variance and use it to derive a 95\% confidence
  interval for \(p\).
\end{enumerate}

Since
\(\sqrt{n}\left(\frac{\bar X_n}{n}-p\right)\overset{d}\rightarrow N(0,p(1-p))\),
the variance of the limiting distribution depends only on \(p\). Note
that \(\frac{X_n}{n}\overset{P}\rightarrow p\) implies
\(\frac{X_n(n-X_n)}{n^2}\overset{P}\rightarrow p(1-p)\). Then by the CLT
and Slutksy's Theorem,
\[\sqrt{n}(X_n/n-p)\sqrt{\frac{n^2}{X_n(n-X_n)}}\overset{d}\rightarrow N(0,1).\]
Therefore, \begin{align*}
&&P\left[-1.96<\sqrt{n}(X_n/n-p)\sqrt{\frac{n^2}{X_n(n-X_n)}}<1.96\right] & \approx 0.95\\
&\implies& P\left[-1.96\frac{\sqrt{X_n(n-X_n)}}{n^{3/2}}<X_n/n-p<1.96\frac{\sqrt{X_n(n-X_n)}}{n^{3/2}}\right] & \approx 0.95\\
&\implies& P\left[X_n/n-1.96\frac{\sqrt{X_n(n-X_n)}}{n^{3/2}}<p<X_n/n+1.96\frac{\sqrt{X_n(n-X_n)}}{n^{3/2}}\right] &\approx 0.95.
\end{align*}

\begin{enumerate}
\def\labelenumi{(\alph{enumi})}
\setcounter{enumi}{1}
\tightlist
\item
  Use the result of problem 5.3(b) to derive a 95\% confidence interval
  for \(p\).
\end{enumerate}

From 5.3(b), we know
\[2\sqrt{n}\left[\sin^{-1}\left(\sqrt{\frac{X_n}{n}}\right)-\sin^{-1}(\sqrt{p})\right]\overset{d}\rightarrow N(0,1).\]
Therefore, \begin{align*}
&&P\left[-1.96 <2\sqrt{n}\left[\sin^{-1}\left(\sqrt{\frac{X_n}{n}}\right)-\sin^{-1}(\sqrt{p})\right]< 1.96\right] & \approx 0.95\\
&\implies&P\left[-\frac{1.96}{2\sqrt{n}} <\sin^{-1}\left(\sqrt{\frac{X_n}{n}}\right)-\sin^{-1}(\sqrt{p})< \frac{1.96}{2\sqrt{n}}\right] & \approx 0.95\\
&\implies&P\left[\sin^{-1}\left(\sqrt{\frac{X_n}{n}}\right)-\frac{1.96}{2\sqrt{n}} <\sin^{-1}(\sqrt{p})<\sin^{-1}\left(\sqrt{\frac{X_n}{n}}\right)+ \frac{1.96}{2\sqrt{n}}\right] & \approx 0.95.
\end{align*} Then consider \[f(x) = \begin{cases}
                                   0 & \text{if } x\le 0 \\
                                   [\sin(x)]^2 & \text{if } 0<x<\pi/2\\
                                   1 & \text{if } x \ge \pi/2.
        \end{cases}\] Thus,
\[P\left[f\left(\sin^{-1}\left(\sqrt{\frac{X_n}{n}}\right)-\frac{1.96}{2\sqrt{n}} \right)<p<f\left(\sin^{-1}\left(\sqrt{\frac{X_n}{n}}\right)+ \frac{1.96}{2\sqrt{n}}\right)\right] \approx 0.95.\]

\begin{enumerate}
\def\labelenumi{(\alph{enumi})}
\setcounter{enumi}{2}
\tightlist
\item
  Evaluate the two confidence intervals in parts (a) and (b) numerically
  for all combinations of \(n \in \{10,100,1000\}\) and
  \(p\in \{.1,.3,.5\}\) as follows: For 1000 realizations of
  \(X\sim \text{bin}(n,p)\), construct both 95\% confidence intervals
  and keep track of how many times (out of 1000) that the confidence
  intervals contain \(p\). Report the observed proportion of successes
  for each \((n,p)\) combination. Does your study reveal any differences
  in the performance of these two competing methods?
\end{enumerate}

The two methods appear to perfrom close to equally.

\begin{Shaded}
\begin{Highlighting}[]
\NormalTok{sim <-}\StringTok{ }\ControlFlowTok{function}\NormalTok{(n, p) \{}
\NormalTok{    x <-}\StringTok{ }\KeywordTok{rbinom}\NormalTok{(}\DecValTok{1000}\NormalTok{, n, p)}
\NormalTok{    ci1 <-}\StringTok{ }\KeywordTok{sum}\NormalTok{(}\KeywordTok{abs}\NormalTok{(}\KeywordTok{sqrt}\NormalTok{(n) }\OperatorTok{*}\StringTok{ }\NormalTok{(x}\OperatorTok{/}\NormalTok{n }\OperatorTok{-}\StringTok{ }\NormalTok{p) }\OperatorTok{*}\StringTok{ }\NormalTok{n}\OperatorTok{/}\KeywordTok{sqrt}\NormalTok{(x }\OperatorTok{*}\StringTok{ }
\StringTok{        }\NormalTok{(n }\OperatorTok{-}\StringTok{ }\NormalTok{x))) }\OperatorTok{<}\StringTok{ }\FloatTok{1.96}\NormalTok{)}
\NormalTok{    ci2 <-}\StringTok{ }\KeywordTok{sum}\NormalTok{(}\KeywordTok{abs}\NormalTok{(}\KeywordTok{sqrt}\NormalTok{(n) }\OperatorTok{*}\StringTok{ }\NormalTok{(}\KeywordTok{asin}\NormalTok{(}\KeywordTok{sqrt}\NormalTok{(x}\OperatorTok{/}\NormalTok{n)) }\OperatorTok{-}\StringTok{ }\KeywordTok{asin}\NormalTok{(}\KeywordTok{sqrt}\NormalTok{(p)))) }\OperatorTok{<}\StringTok{ }
\StringTok{        }\FloatTok{0.98}\NormalTok{)}
    \KeywordTok{c}\NormalTok{(ci1, ci2)}\OperatorTok{/}\DecValTok{1000}
\NormalTok{\}}

\NormalTok{nseq <-}\StringTok{ }\KeywordTok{c}\NormalTok{(}\DecValTok{10}\NormalTok{, }\DecValTok{100}\NormalTok{, }\DecValTok{1000}\NormalTok{)}
\NormalTok{pseq <-}\StringTok{ }\KeywordTok{c}\NormalTok{(}\FloatTok{0.1}\NormalTok{, }\FloatTok{0.3}\NormalTok{, }\FloatTok{0.5}\NormalTok{)}
\NormalTok{results <-}\StringTok{ }\KeywordTok{matrix}\NormalTok{(}\OtherTok{NA}\NormalTok{, }\DecValTok{9}\NormalTok{, }\DecValTok{5}\NormalTok{)}
\KeywordTok{colnames}\NormalTok{(results) <-}\StringTok{ }\KeywordTok{c}\NormalTok{(}\StringTok{"n"}\NormalTok{, }\StringTok{"p"}\NormalTok{, }\StringTok{"coverageA"}\NormalTok{, }\StringTok{"coverageB"}\NormalTok{, }
    \StringTok{"ratio"}\NormalTok{)}
\NormalTok{it =}\StringTok{ }\DecValTok{0}
\ControlFlowTok{for}\NormalTok{ (i }\ControlFlowTok{in} \DecValTok{1}\OperatorTok{:}\DecValTok{3}\NormalTok{) \{}
    \ControlFlowTok{for}\NormalTok{ (j }\ControlFlowTok{in} \DecValTok{1}\OperatorTok{:}\DecValTok{3}\NormalTok{) \{}
\NormalTok{        it =}\StringTok{ }\NormalTok{it }\OperatorTok{+}\StringTok{ }\DecValTok{1}
\NormalTok{        coverage <-}\StringTok{ }\KeywordTok{sim}\NormalTok{(nseq[i], pseq[j])}
\NormalTok{        results[it, }\DecValTok{1}\NormalTok{] <-}\StringTok{ }\NormalTok{nseq[i]}
\NormalTok{        results[it, }\DecValTok{2}\NormalTok{] <-}\StringTok{ }\NormalTok{pseq[j]}
\NormalTok{        results[it, }\DecValTok{3}\NormalTok{] <-}\StringTok{ }\NormalTok{coverage[}\DecValTok{1}\NormalTok{]}
\NormalTok{        results[it, }\DecValTok{4}\NormalTok{] <-}\StringTok{ }\NormalTok{coverage[}\DecValTok{2}\NormalTok{]}
\NormalTok{        results[it, }\DecValTok{5}\NormalTok{] <-}\StringTok{ }\NormalTok{coverage[}\DecValTok{1}\NormalTok{]}\OperatorTok{/}\NormalTok{coverage[}\DecValTok{2}\NormalTok{]}
\NormalTok{    \}}
\NormalTok{\}}
\KeywordTok{kable}\NormalTok{(results)}
\end{Highlighting}
\end{Shaded}

\begin{longtable}[]{@{}rrrrr@{}}
\toprule
n & p & coverageA & coverageB & ratio\tabularnewline
\midrule
\endhead
10 & 0.1 & 0.671 & 0.666 & 1.0075075\tabularnewline
10 & 0.3 & 0.845 & 0.970 & 0.8711340\tabularnewline
10 & 0.5 & 0.871 & 0.871 & 1.0000000\tabularnewline
100 & 0.1 & 0.927 & 0.955 & 0.9706806\tabularnewline
100 & 0.3 & 0.946 & 0.946 & 1.0000000\tabularnewline
100 & 0.5 & 0.943 & 0.943 & 1.0000000\tabularnewline
1000 & 0.1 & 0.952 & 0.948 & 1.0042194\tabularnewline
1000 & 0.3 & 0.954 & 0.955 & 0.9989529\tabularnewline
1000 & 0.5 & 0.955 & 0.955 & 1.0000000\tabularnewline
\bottomrule
\end{longtable}

\hypertarget{exercise-5.6}{%
\subsubsection{Exercise 5.6}\label{exercise-5.6}}

Suppose that \(X_1, X_2,...\) are independent and identically
distributed Normal \((0,\sigma^2)\) random variables.

\begin{enumerate}
\def\labelenumi{(\alph{enumi})}
\tightlist
\item
  Based on the result of Example 5.7, Give an approximate test at
  \(\alpha = .05\) for \(H_0:\sigma^2=\sigma_0^2\) vs
  \(H_a:\sigma^2\ne \sigma_0^2.\)
\end{enumerate}

From Example 5.7, we know
\[\sqrt{n}\left[\log\left(\frac{1}{n}\sum_{i=1}^n X_i^2\right)-\log(\sigma^2)\right]\overset{d}\rightarrow N(0,2).\]
Then,
\[T_n = \frac{\sqrt{n}\left[\log\left(\frac{1}{n}\sum_{i=1}^n X_i^2\right)-\log(\sigma_0^2)\right]}{\sqrt{2}}\overset{d}\rightarrow N(0,1).\]
For a size \(\alpha = 0.05\) test, we reject if \(|T_n|>1.96.\)

\begin{enumerate}
\def\labelenumi{(\alph{enumi})}
\setcounter{enumi}{1}
\tightlist
\item
  For \(n=25\), estimate the true level of the test in part (a) for
  \(\sigma_0^=1\) by simulating 5000 samples of size \(n=25\) from the
  null distribution. Report the proportion of cases in which you reject
  the null hypothesis according to your test (ideally, this proportion
  will be about .05).
\end{enumerate}

\begin{Shaded}
\begin{Highlighting}[]
\NormalTok{sim2 <-}\StringTok{ }\ControlFlowTok{function}\NormalTok{(n, sigsq, samps) \{}
\NormalTok{    reject <-}\StringTok{ }\KeywordTok{rep}\NormalTok{(}\OtherTok{NA}\NormalTok{, samps)}
    \ControlFlowTok{for}\NormalTok{ (i }\ControlFlowTok{in} \DecValTok{1}\OperatorTok{:}\NormalTok{samps) \{}
\NormalTok{        Xn <-}\StringTok{ }\KeywordTok{rnorm}\NormalTok{(}\DataTypeTok{n =}\NormalTok{ n, }\DataTypeTok{mean =} \DecValTok{0}\NormalTok{, }\DataTypeTok{sd =} \KeywordTok{sqrt}\NormalTok{(sigsq))}
\NormalTok{        Tn <-}\StringTok{ }\NormalTok{(}\KeywordTok{sqrt}\NormalTok{(n) }\OperatorTok{*}\StringTok{ }\NormalTok{(}\KeywordTok{log}\NormalTok{((}\DecValTok{1}\OperatorTok{/}\NormalTok{n) }\OperatorTok{*}\StringTok{ }\KeywordTok{sum}\NormalTok{(Xn}\OperatorTok{^}\DecValTok{2}\NormalTok{)) }\OperatorTok{-}\StringTok{ }
\StringTok{            }\KeywordTok{log}\NormalTok{(sigsq)))}\OperatorTok{/}\KeywordTok{sqrt}\NormalTok{(}\DecValTok{2}\NormalTok{)}
        \ControlFlowTok{if}\NormalTok{ (Tn }\OperatorTok{<}\StringTok{ }\DecValTok{0}\NormalTok{) \{}
\NormalTok{            pvalue <-}\StringTok{ }\DecValTok{2} \OperatorTok{*}\StringTok{ }\KeywordTok{pnorm}\NormalTok{(}\DataTypeTok{q =}\NormalTok{ Tn, }\DataTypeTok{lower.tail =}\NormalTok{ T)}
\NormalTok{        \} }\ControlFlowTok{else}\NormalTok{ \{}
\NormalTok{            pvalue <-}\StringTok{ }\DecValTok{2} \OperatorTok{*}\StringTok{ }\KeywordTok{pnorm}\NormalTok{(}\DataTypeTok{q =}\NormalTok{ Tn, }\DataTypeTok{lower.tail =}\NormalTok{ F)}
\NormalTok{        \}}
\NormalTok{        reject[i] <-}\StringTok{ }\NormalTok{(pvalue }\OperatorTok{<}\StringTok{ }\FloatTok{0.05}\NormalTok{)}
\NormalTok{    \}}
    \KeywordTok{sum}\NormalTok{(reject)}\OperatorTok{/}\NormalTok{samps}
\NormalTok{\}}
\KeywordTok{sim2}\NormalTok{(}\DecValTok{25}\NormalTok{, }\DecValTok{1}\NormalTok{, }\DecValTok{5000}\NormalTok{)}
\end{Highlighting}
\end{Shaded}

\begin{verbatim}
## [1] 0.0626
\end{verbatim}

\hypertarget{exercise-5.8}{%
\subsubsection{Exercise 5.8}\label{exercise-5.8}}

Assume \((X_1,Y_1),...,(X_n,Y_n)\) are independent and identically
distributed from some bivariate normal distribution. Let \(\rho\) denote
the population correlation coefficient and \(r\) the sample correlation
coefficient.

\begin{enumerate}
\def\labelenumi{(\alph{enumi})}
\tightlist
\item
  Describe a test of \(H_0:\rho=0\) against \(H_1:\rho\ne 0\) based on
  the fact that \[\sqrt{n}[f(r)-f(\rho)]\overset{d}\rightarrow N(0,1),\]
  where \(f(x)\) is a Fisher's transformation
  \(f(x) = (1/2)\log [(1+x)/(1-x)]\). Use \(\alpha = .05\).
\end{enumerate}

Consider
\[T_n = \sqrt{n}[f(r)-f(\rho_0)]\overset{d}\rightarrow N(0,1).\] For a
size \(\alpha = 0.05\) test, we reject if \(|T_n|>1.96.\)

\begin{enumerate}
\def\labelenumi{(\alph{enumi})}
\setcounter{enumi}{1}
\tightlist
\item
  Based on 5000 repetitions each, estimate the actual level for this
  test in the case when
  \(\text{E}(X_i)=\text{E} (Y_i)=0, \text{Var}(X_i)=\text{Var}(Y_i)=1,\)
  and \(n\in\{3,5,10,20\}\).
\end{enumerate}

The level of the test improves as the sample increases.

\begin{Shaded}
\begin{Highlighting}[]
\NormalTok{fisher <-}\StringTok{ }\ControlFlowTok{function}\NormalTok{(x) \{}
    \KeywordTok{log}\NormalTok{((}\DecValTok{1} \OperatorTok{+}\StringTok{ }\NormalTok{x)}\OperatorTok{/}\NormalTok{(}\DecValTok{1} \OperatorTok{-}\StringTok{ }\NormalTok{x))}\OperatorTok{/}\DecValTok{2}
\NormalTok{\}}
\NormalTok{rtest <-}\StringTok{ }\ControlFlowTok{function}\NormalTok{(n) \{}
\NormalTok{    z <-}\StringTok{ }\KeywordTok{array}\NormalTok{(}\KeywordTok{rnorm}\NormalTok{(}\DecValTok{10000} \OperatorTok{*}\StringTok{ }\NormalTok{n), }\KeywordTok{c}\NormalTok{(}\DecValTok{5000}\NormalTok{, n, }\DecValTok{2}\NormalTok{))}
\NormalTok{    r <-}\StringTok{ }\KeywordTok{apply}\NormalTok{(z, }\DecValTok{1}\NormalTok{, }\ControlFlowTok{function}\NormalTok{(x) }\KeywordTok{cor}\NormalTok{(x[, }\DecValTok{1}\NormalTok{], x[, }\DecValTok{2}\NormalTok{]))}
\NormalTok{    Tn <-}\StringTok{ }\KeywordTok{sqrt}\NormalTok{(n) }\OperatorTok{*}\StringTok{ }\NormalTok{(}\KeywordTok{fisher}\NormalTok{(r) }\OperatorTok{-}\StringTok{ }\KeywordTok{fisher}\NormalTok{(}\DecValTok{0}\NormalTok{))}
    \KeywordTok{sum}\NormalTok{(}\KeywordTok{abs}\NormalTok{(Tn) }\OperatorTok{>}\StringTok{ }\FloatTok{1.96}\NormalTok{)}
\NormalTok{\}}
\KeywordTok{sapply}\NormalTok{(}\KeywordTok{c}\NormalTok{(}\DecValTok{3}\NormalTok{, }\DecValTok{5}\NormalTok{, }\DecValTok{10}\NormalTok{, }\DecValTok{20}\NormalTok{), rtest)}\OperatorTok{/}\DecValTok{5000}
\end{Highlighting}
\end{Shaded}

\begin{verbatim}
## [1] 0.3872 0.1858 0.1060 0.0724
\end{verbatim}

\hypertarget{exercise-5.9}{%
\subsubsection{Exercise 5.9}\label{exercise-5.9}}

Suppose that \(X\) and \(Y\) are jointly distributed such that \(X\) and
\(Y\) are Bernoulli \((1/2)\) random variables with \(P(XY=1)=\theta\)
for \(\theta\in (0,1/2)\). Let \((X_1,Y_1), (X_2,Y_2),...\) be
independent and identically distributed with \((X_i, Y_i)\) distributed
as \((X,Y)\).

\begin{enumerate}
\def\labelenumi{(\alph{enumi})}
\tightlist
\item
  Find the asymptotic distribution of
  \(\sqrt{n}[(\bar X_n, \bar Y_n) - (1/2, 1/2)].\)
\end{enumerate}

Since \(X\sim\text{Bern}(1/2)\) and \(Y\sim\text{Bern}(1/2)\), we know
\(\sigma_x^2 = \sigma_y^2 = 1/4\). Since \(P(XY=1)=\theta,\) we know
\(\sigma_xy = E[XY]-E[X]E[Y]=\theta-1/4\) and \(\rho = 4\theta - 1.\)
Then
\[\sqrt{n}\left[\begin{pmatrix}\bar X_n \\ \bar Y_n\end{pmatrix}-\begin{pmatrix}1/2 \\ 1/2\end{pmatrix}\right]\overset{d}\rightarrow N_2\left(\begin{pmatrix}0 \\ 0\end{pmatrix}, \begin{pmatrix} 1/4 & \theta-1/4 \\ \theta - 1/4 & 1/4\end{pmatrix}\right).\]

\begin{enumerate}
\def\labelenumi{(\alph{enumi})}
\setcounter{enumi}{1}
\tightlist
\item
  If \(r_n\) is the sample correlation coefficient for a sample of size
  \(n\), find the asymptotic distribution of \(\sqrt{n}(r_n-\rho).\)
\end{enumerate}

We know,
\(\sqrt{n}(r_n-\rho) \overset{d}\rightarrow N(0, A\Sigma^{*} A^T)\)
where

\[A = \left(\frac{-\sigma_{xy}}{2\sigma_x^3\sigma_y}, \frac{-\sigma_{xy}}{2\sigma_x\sigma_y^3}, \frac{1}{\sigma_x\sigma_y}\right) =\left(2(1-4\theta), 2(1-4\theta), 4\right)\]
and \[\Sigma^{*} = \begin{pmatrix}
cov(X,X) & cov(X,Y) & cov(X,XY)\\
cov(Y,X) & cov(Y,Y) & cov(Y,XY)\\
cov(XY,X) & cov(XY,Y) & cov(XY,XY)
\end{pmatrix} =\begin{pmatrix}
1/4 & \theta - 1/4 & \theta/2\\
\theta - 1/4 & 1/4 & \theta/2\\
\theta/2 & \theta/2 & \theta(1-\theta)
\end{pmatrix}\]

since \(X^2 = X\) and \(Y^2=Y\). Therefore,

\[A\Sigma^{*}A^T = 128\theta^3-144\theta^2+40\theta.\] Thus, since
\(\theta = \frac{\rho+1}{r},\)
\[\sqrt{n}(r_n-\rho)\overset{d}\rightarrow N(0,128\theta^3-144\theta^2+40\theta)\overset{d}=N(0, 2\rho^3-3\rho^2-2\rho+3).\]

\begin{enumerate}
\def\labelenumi{(\alph{enumi})}
\setcounter{enumi}{2}
\item
  Find a variance stabilizing transformation for \(r_n\).
\item
  Based on your answer to part (c), construct a 95\% confidence interval
  for \(\theta\).
\item
  For each combination of \(n\in\{5,20\}\) and
  \(\theta \in \{.05, .25, .45\}\), estimate the true coverage
  probability of the confidence interval in part (d) by simulating 5000
  samples and the corresponding confidence intervals. One problem you
  will face is that in some samples, the sample correlation coefficient
  is undefined because with positive probability each of the \(X_i\) or
  \(Y_i\) will be the same. In such cases, consider the confidence
  interval to be undefined and the true parameter therefore not
  contained therein.

  \textbf{Hint:} To generate a sample of \((X,Y)\), first simulate the
  \(X\)'s from their marginal distribution, then simulate the \(Y\)'s
  according to the conditional distribution of \(Y\) given \(X\). TO
  obtain this conditional distribution, find \(P(Y=1|X=1)\) and
  \(P(Y=1|X=0).\)
\end{enumerate}


\end{document}
